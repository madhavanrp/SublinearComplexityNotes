\documentclass[11pt]{article}
\usepackage{fullpage}
\usepackage{amsmath}
\usepackage{amssymb}
\usepackage{algorithm}
\usepackage{algpseudocode}

\usepackage{amsmath}
\usepackage{amsthm}
\usepackage{amssymb}
\usepackage{epsfig}
\newtheorem{theorem}{Theorem}
\newtheorem{lemma}{Lemma}
\newtheorem{claim}{Claim}
\newtheorem{proposition}{Proposition}
\newtheorem{corollary}{Corollary}
\theoremstyle{definition} \newtheorem{definition}{Definition}
\newtheorem{fact}{Fact}
\newenvironment{Proof}{{\it Proof. }}{$\Box$\\}
\newcommand{\numberP}{{\mathchoice{\raisebox{1pt}
        {$\displaystyle\#$}{\rm P}}{\raisebox{1pt}
        {$\textstyle\#$}{\rm P}}{\raisebox{1pt}
        {$\scriptstyle\#$}{\rm P}}{\raisebox{1pt}
        {$\scriptscriptstyle\#$}{\rm P}}}}
\newcommand{\sharpP}{\numberP}
\newcommand{\Hprime}{${\mathrm H}^{'}$}
\newcommand{\mo}{\leq^p_m}
\newcommand{\tr}{\leq^p_T}
\newcommand{\ttr}{\leq^p_{tt}}
\newcommand{\btt}{\leq^p_{btt}}
\newcommand{\onett}{\leq^p_{1tt}}
\newcommand{\twott}{\leq^p_{2tt}}
\newcommand{\pos}{\leq^p_{pos}}
\newcommand{\postt}{\leq^p_{postt}}
\newcommand{\ppcomp}{\P_{\pcomp}}
\newcommand{\DistS}{{\rm Dist\Sigma_2}}
\newcommand{\DistPNP}{{\rm Dist\P^{\NP}}}
\newcommand{\AVTIME}{{\rm AVTIME}}
\newcommand{\ppoly}{{\P/{\rm poly}}}

\newcommand{\ignore}[1]{}

\newcommand{\mathmake}[1]{{\mathchoice%
   {\raisebox{0pt}{$\displaystyle #1$}}
   {\raisebox{0pt}{$\textstyle #1$}}
   {\raisebox{0pt}{$\scriptstyle #1$}}
   {\raisebox{0pt}{$\scriptscriptstyle #1$}}}}

\newcommand{\mathraise}[2]{{\mathchoice%
   {\raisebox{1pt}{$\displaystyle #1$}{#2}}
   {\raisebox{1pt}{$\textstyle #1$}{#2}}
   {\raisebox{1pt}{$\scriptstyle #1$}{#2}}
   {\raisebox{1pt}{$\scriptscriptstyle #1$}{#2}}}}

\newcommand{\newfontobj}[2]{
  \newcommand{#1}[1]{
    \expandafter\def\csname##1\endcsname{{#2 ##1}}}}

\newfontobj{\class}{\rm}
\newfontobj{\lang}{\bf}
\newfontobj{\oper}{\rm}


\class{PSPACE}          % Language classes have to implement
\class{AM}              % a clean way of doing the complement classes
\class{MA}
\class{NP}
\class{P}
\class{RP}
\class{BPP}
\class{advBPP}
\class{DTIME}
\class{ZPP}
\class{EXPSPACE}
\class{PH}
\class{IP}
\class{AVP}
\newcommand{\AP}{{\rm Average\mbox{-}P}}
\class{DistNP}
\class{AZPP}
\class{AVE}
\class{DistE}
\class{DistEXP}
\class{E}
\class{NE}
\class{UE}
\class{EXP}
\class{pcomp}
\class{NEXP}
\class{UP}
\class{DTIME}
\class{NTIME}
\class{R}
\newcommand{\CoNP}{{\mbox {\rm co-NP}}}
\newcommand{\CoUP}{{\mbox {\rm co-UP}}}
\newcommand{\CoNE}{{\mbox{\rm co-NE}}}
\newcommand{\SAT}{{\mbox SAT}}



\newcommand{\norm}[1]{\left\lVert#1\right\rVert}
\begin{document}
\begin{center}
\begin{tabular} {|lcl|}
\hline
& Com S 631: Dimensionality Reduction & \\
Lectures 14,15 & & Scribe: Madhavan \\
\hline
\end{tabular}
\end{center}
%\maketitle


\section{Dimensionality Reduction}

Let $S$ be a set of data points in high dimension space i.e. $S\subseteq \mathbb{R}^m$. The objective is to map $S$ to a low dimension space while preserving certain \emph{properties} of the data points. For example, consider the Euclidean distance of two points. Let $x,y\in \mathbb{R}^m$ and let $f:\mathbb{R}^m\to \mathbb{R}^p$ where $p<<m$. We would like to design $f$ such that $\norm{x-y}_2 \approx f(x)-f(y)$. 
\smallskip
\subsection{AMS algorithm for estimating F2}
Let us revisit the AMS algorithm for estimating the second frequency moment. Consider a stream $Z$ over $U=\{1..m\}$. Let $f_i$ be the number of times that element $i$ appears. The second frequency moment $F_2(Z)=\sum_{i=1}^{m}f_i^2$. Let $F=<f_1, f_2, .. , f_m> $ be a vector in $\mathbb{R}^m$. Equivalently, define $F_2(Z)=\norm{F}_2^2$. 

\begin{algorithm}
\caption{AMS Algorithm With Repititions}
 \hspace*{\algorithmicindent} \textbf{Input} Stream $Z$ over $U=\{1..m\}$ \\
 \hspace*{\algorithmicindent} \textbf{Output} Estimate of $F_2(Z)$
\label{alg:amsrep}
\begin{algorithmic}[1]
\Procedure{AMS }{}
    \State For $i=1..k$, generate 4-wise independent hash functions $h_i:U\to \{+1,-1\}$
    \For { $x \in S$ }
        \For{$i=1$ to $k$} 
            \State $c_i=c_i + h(x)$
        \EndFor
    \EndFor
    \State $\hat{F_2}(Z)=\frac{\sum_{i=1}^{k}c_i^2}{k}$
\State \Return $\hat{F_2}(Z)$
\EndProcedure
\end{algorithmic}
\end{algorithm}

\textbf{Idea:} The above AMS algorithm is performing a dimensionality reduction. We will show that the estimate $\hat{F_2}(Z)$ is close to $F_2(Z)$ with high probability. If that is the case then we are mapping a vector $F=<f_1, f_2.., f_m>\to <\frac{c_1}{\sqrt{k}}, \frac{c_2}{\sqrt{k}}.. \frac{c_k}{\sqrt{k}}>=C$.

\begin{theorem}\label{thm:ams}
If $k=O(\frac{1}{\epsilon^2})$ then $P\left[\norm{C}_2^2 \in (1 \pm \epsilon) \norm{F}_2^2 \right] \geq 1- \delta \geq \frac{2}{3}$. Here, $\delta=\frac{1}{3}$
\end{theorem}

Consider the hash function $h_i:U\to \{+1,-1\}$. Define the random variable $X_{ij}=1$ if $h_i(j)=1$, otherwise $X_{ij}=-1$. Therefore, we define $c_1=\sum_{j=1}^{m}X_{1j}f_j$, $c_2=\sum_{j=1}^{m}X_{2j}f_j$ and so on. We can it as follows: \\
\[
\begin{bmatrix}
 X_{11} & X_{12} & \cdots & X_{1m} \\
 \vdots & \vdots & \vdots & \vdots \\
 X_{k1} & X_{k2} & \cdots & X_{km} 
\end{bmatrix}
\begin{bmatrix}
 f_1 \\
 \vdots\\
 f_m
\end{bmatrix}
=
\begin{bmatrix}
 c_1 \\
 \vdots\\
 c_k
\end{bmatrix}
\]

The $(k\times m)$ matrix is called as the AMS sketch. Observe that the vector $F$ can be expressed as the difference between two vectors $a$ and $b$. In other words, if $a,b\in \mathbb{R}^m$ then $(a-b)\in \mathbb{R}^m$.
\smallskip
Consider $S\subseteq \mathbb{R}^m$. Using AMS with $k$ hash functions, we are able to claim $P\left[\norm{C}_2^2 \in (1 \pm \epsilon) \norm{F}_2^2 \right] \geq 1- \delta$. This event corresponds to the distance between one pair of vectors. Can we bound the union of the events that correspond to the distance between every pair of vectors in $S$. 

    \begin{equation}
    \begin{split}
    P\left[\forall x, y\in S, \norm{x-y}_2^2 \notin (1 \pm \epsilon) \norm{x-y}_2^2 \right] &\leq n(n-1) P\left[x,y\in S, \norm{x-y}_2^2 \notin (1 \pm \epsilon) \norm{x-y}_2^2 \right] (Union Bound)\\
    &\leq n^2 \times O\left(\frac{1}{\epsilon^2}\right) \times \frac{1}{k}
    \end{split}
    \end{equation}
To bound the above probability by $\delta$, we need $k=n^2O\left(\frac{1}{\epsilon^2}\right)$. Therefore, $\delta=O\left(\frac{1}{\delta\epsilon^2}\right)$. We can reduce it to $O\left(log(\frac{1}{\delta})\frac{1}{\epsilon^2}\right)$, thereby reducing the number of hash functions.
\section{Johnson Lindenstrauss Theorem}
The Johnson-Lindenstrauss theorem states that there exists a mapping $f:\mathbb{R}^m\to \mathbb{R}^k$ such that the Euclidean distance of any two points in $\mathbb{R}^m$ is approximately preserved. We will use the following lemma to arrive at a precise definition and proof of the theorem.

\begin{lemma}
\label{lemma:jllemma}
Let $A$ be a $k\times m$ matrix where each entry is independently drawn from $N(0,1)$. Then, $\forall x\in \mathbb{R}^m$, $P\left[\norm{Ax}_2\in (1+\epsilon) \norm{x}_2 \right]>1-\delta$ where $k=O\left(log(\frac{1}{\delta})\frac{1}{\epsilon^2}\right)$
\end{lemma}

Lemma ~\ref{lemma:jllemma} is called the distributional JL-Lemma. We use this lemma to state and prove the JL theorem.
\begin{theorem}
\label{thm:jltheorem}
Let $S\subseteq \mathbb{R}^m$ with n points. There exists $k\times m$ matrix $A$ such that $\forall x,y\in S$,$\norm{Ax - Ay}_2\in (1\pm \epsilon) \norm{x - y}_2$ where $k=O\left(\frac{1}{\epsilon^2}logn\right)$
\end{theorem}

\begin{proof}
Let $A$ be the Gaussian Matrix from JL Lemma. Since $|S|=n$, there are $n\choose 2$ pairs of points in $S$. We have: \\
\begin{equation}
P\left[\forall x, y, \norm{Ax-Ay}_2\in (1\pm\epsilon) \norm{x-y}_2 \right] = 1- P\left[\exists x, y, s.t \norm{Ax-Ay}_2\notin (1\pm\epsilon) \norm{x-y}_2 \right]\\
\end{equation}
\begin{equation}
\begin{split}
P\left[\exists x, y, s.t \norm{Ax-Ay}_2\notin (1\pm\epsilon) \norm{x-y}_2 \right] &\leq \sum_{x,y} P\left[\norm{Ax-Ay}_2\notin (1\pm\epsilon) \norm{x-y}_2 \right]\\
&\leq \sum_{x,y} \delta \text{(From JL Lemma)}\\ 
&\leq n^2 \delta = \frac{1}{2}
\end{split}
\end{equation}
The final equality is true if $\delta=\frac{1}{2n^2}$. 
\begin{equation}
P\left[\forall x, y, \norm{Ax-Ay}_2\in (1\pm\epsilon) \norm{x-y}_2 \right] \geq \frac{1}{2}
\end{equation}
Therefore, there exists a matrix $A$ such that $P\left[\forall x, y, \norm{Ax-Ay}_2\in (1\pm\epsilon) \norm{x-y}_2 \right]$ is at least half. 
\end{proof}

We will review probability distribution over reals and then proceed to prove JL Lemma.
\subsection{Continuous Random Variables, Probability Density Function}
A random variable $X$ is continuous if there exists a function $f$ such that:
\begin{equation}
Pr\left[a \leq X \leq b\right]=\int_{a}^{b}f(x)dx
\end{equation}
$f$ is called the Probability Distribution Function (PDF). $f$ satisfies the following properties:
\begin{enumerate}
\item $f(x)>0, \forall x$ 
\item $\int_{-\infty}^{\infty}f(x)dx=1$
\end{enumerate}

If $X$ is a continuous random variable that takes the value returned by the PDF, its expectation is defined as:
\begin{equation}
E[X] = \int_{-\infty}^{\infty}f(x)x dx
\end{equation}

\textbf{Gaussian Distribution}
The Gaussian distribution (a.k.a Normal Distribution) is defined by the probability density function:
\begin{equation}
f(x) = \frac{1}{\sqrt{2\pi}\sigma}e^{\frac{-(x-\mu)^2}{2\sigma^2}}
\end{equation}
Where $\mu$ is the mean (or expectation), $\sigma$ is the standard deviation. The Gaussian distribution is written as $N(\mu,\sigma^2)$. If a random variable $X$ is drawn according to the PDF of a Gaussian distribution with mean (or expectation) $\mu$ and standard deviation $\sigma$, we write it as $X\sim N(\mu,\sigma^2)$. The distribution has the following properties:
\begin{enumerate}
\item The variance $Var(X)=\sigma^2$.
\item If $X\sim N(0,\sigma^2)$ $Var(X)= E[X^2] - E[X]^2 = E[X^2]$
\item If $X\sim N_1(\mu_1,\sigma_1^2)$ and $Y\sim N_2(\mu_2,\sigma_2^2)$ are independently drawn then:
    \begin{itemize}
        \item $X+Y \sim N(\mu_1 + \mu_2, \sigma_1^2 + \sigma_2^2)$
    \end{itemize}
\item If $c$ is a constant and $X\sim N(\mu,\sigma^2)$ then $cX\sim N(\mu, c^2\sigma^2)$
\end{enumerate}

\subsection{Moment Generating Function}
The Moment Generating Function of a random variable $X$, where $\lambda \in \mathbb{R}$ :
\begin{equation}
\begin{split}
MGF(X) &= E[e^{\lambda X}]\\
&= E[1 + \lambda X + \frac{\lambda^2X^2}{2!} + ...]\\
\end{split}
\end{equation}
\begin{equation}
\begin{split}
Pr[X>t] &= Pr[e^{\lambda X}>e^{\lambda t}]\\
&\leq \frac{E[e^{\lambda X}]}{e^{\lambda t}} \text{(Using Markov)}\\
&= \frac{MGF(X)}{e^{\lambda t}}
\end{split}
\end{equation}
\begin{equation}
\begin{split}
MGF(X) &= E[e^{\lambda X}] \\
&= \int_{-\infty}^{\infty} e^{\lambda x}e^{\frac{-x^2}{2\sigma^2}}\frac{1}{\sqrt{2\pi}\sigma} dx\\
&= \int_{-\infty}^{\infty} \frac{1}{\sqrt{2\pi}\sigma} e^{\frac{-x^2 + 2\sigma^2\lambda x}{2\sigma^2}} dx
\end{split}
\end{equation}

\begin{equation}
\begin{split}
\frac{-x^2 + 2\sigma^2\lambda x}{2\sigma^2} &= \frac{-x^2 + 2\sigma^2\lambda x -\sigma^4\lambda^2 + \sigma^4\lambda^2}{2\sigma^2} \\
&= \frac{-(x-\sigma^2\lambda)^2}{2\sigma^2} + \frac{\sigma^2\lambda^2}{2}
\end{split}
\end{equation}

Therefore,
\begin{equation}
\begin{split}
MGF(X) &= \int_{-\infty}^{\infty} \frac{1}{\sqrt{2\pi}\sigma} e^{\frac{-(x-\sigma^2\lambda)^2}{2\sigma^2}} e^{\frac{\sigma^2\lambda^2}{2}} dx \\
&= e^{\frac{\sigma^2\lambda^2}{2}} \int_{-\infty}^{\infty} \frac{1}{\sqrt{2\pi}\sigma} e^{\frac{-(x-\sigma^2\lambda)^2}{2\sigma^2}} dx\\
&= e^{\frac{\sigma^2\lambda^2}{2}} \int_{-\infty}^{\infty} \text{PDF of} N(\sigma^2\lambda, \sigma^2) dx\\
&= e^{\frac{\sigma^2\lambda^2}{2}}
\end{split}
\end{equation}
Therefore, we have $MGF(X) = e^{\frac{\sigma^2\lambda^2}{2}}$. Suppose we draw $X\sim N(0, \sigma^2)$ and we want to find the probability $Pr[X>t]$?
\begin{equation}
\begin{split}
Pr[X>t] &= Pr[e^{\lambda X}>e^{\lambda t}] \\
&\leq \frac{MGF(X)}{e^{\lambda t}} \text{(Markov bound)} \\
&= \frac{e^{\frac{\sigma^2\lambda^2}{2}}}{e^{\lambda t}} \\
&= e^{\frac{\sigma^2\lambda^2}{2} - \lambda t}
\end{split}
\end{equation}
The above function is minimized when $\lambda=\frac{t}{\sigma^2}$. Therefore, $Pr[X>t]\leq e^{\frac{-t^2}{2\sigma^2}}$. Similarly, $Pr[-X>t]\leq e^{\frac{-t^2}{2\sigma^2}}$. This leads to the following theorem.

\begin{theorem}
$Pr[|X|>t]\leq 2e^{\frac{-t^2}{2\sigma^2}}$
\end{theorem}

\subsection{Chi-Squared Distribution}
$\chi^2$ distribution is the with $k$ degrees of freedom is the sum of the squares of $k$ independent random variables drawn a Gaussian distribution with $\mu=0, \sigma=1$. Formally, if $X_1. X_2...X_k$ are independent random variables drawn from $N(0,1)$ then the $\chi^2$ distribution is $Z$ where:
\begin{equation}
Z=\sum_{i=1}^{k} X_i^2
\end{equation}

\begin{theorem}
Let $Z=\sum_{i=1}^{k} X_i^2$ where $X_1, X_2..X_k$ are random variables drawn independently from $N(0,1)$. Then, $Pr[|Z-k|\geq \epsilon k] \leq 2 e^{-k(\epsilon^2 - d\epsilon^3)}$ for some $d>0$.
\end{theorem}
\begin{proof}
The expectation of $Z$, $E[Z] = \sum_{i=1}^{k} E[X_i^2] = k$. This is because $E[X_i^2] = \sigma^2 = 1$.
\begin{equation}
\begin{split}
Pr[Z> k + \epsilon k] &= Pr[e^{\lambda Z}> e^{\lambda(k+\epsilon k)}] \\
&\leq \frac{E[e^{\lambda Z}]}{e^{\lambda(k+\epsilon k)}} \text{(Using Markov)}
\end{split}
\end{equation}
\begin{equation}
\begin{split}
E[e^{\lambda Z}] &= E[e^{\lambda(X_1^2 + X_2^2 + ... + X_k^2)}] \\
&= E[e^{\lambda X_1^2}e^{\lambda X_2^2}...e^{\lambda X_k^2}]\\
&= E[e^{\lambda X_1^2}] \times E[e^{\lambda X_2^2}] ... \times E[e^{\lambda X_k^2}]
\end{split}
\end{equation}
The last equality comes from the fact that if the random variables are independent, the expectation of the product is the product of expectations.

\begin{equation}
\begin{split}
E[e^{\lambda X^2}] &= \int_{-\infty}^{\infty} e^{\lambda X^2} \frac{1}{\sqrt{2\pi}} e^{\frac{-x^2}{2}} dx \\
&= \int_{-\infty}^{\infty} \frac{1}{\sqrt{2\pi}} e^{\frac{2\lambda x^2-x^2}{2}} dx
\end{split}
\end{equation}

\begin{equation}
\begin{split}
e^{\frac{2\lambda x^2-x^2}{2}} &= e^{\frac{-x^2(1-2\lambda)}{2}} \\
&= e^{\frac{-x^2}{2(1-2\lambda)^{-1}}}
\end{split}
\end{equation}

\begin{equation}
\begin{split}
E[e^{\lambda X^2}] &= \int_{-\infty}^{\infty} \frac{1}{\sqrt{2\pi}} e^{\frac{-x^2}{2(1-2\lambda)^{-1}}} dx \\
&= \int_{-\infty}^{\infty} \frac{\sqrt{(1-2\lambda)^{-1}}}{\sqrt{2\pi}\sqrt{(1-2\lambda)^{-1}}} e^{\frac{-x^2}{2(1-2\lambda)^{-1}}} dx \\
&= \frac{1}{\sqrt{(1-2\lambda)}} \int_{-\infty}^{\infty} \text{PDF of } N(0,(1-2\lambda)^{-1}) dx\\
&= \frac{1}{\sqrt{(1-2\lambda)}}
\end{split}
\end{equation}

Plugging this back in, we get,
\begin{equation}
E[e^{\lambda X^2}] = \left(\frac{1}{\sqrt{(1-2\lambda)}}\right)^k = \left(\frac{1}{1-2\lambda}\right)^\frac{k}{2}
\end{equation}

\begin{equation}
\frac{MGF(Z)}{e^{\lambda(k+\epsilon k)}} \leq \left(\frac{1}{1-2\lambda}\right)^\frac{k}{2} \times \frac{1}{e^{\lambda(k+\epsilon k)}}
\end{equation}

The expression is minimized when $\lambda=\frac{\epsilon}{2(1+\epsilon)}$. Then, $1-2\lambda=\frac{1}{1+\epsilon}$. Therefore,

\begin{equation}
\begin{split}
Pr[Z-k\geq \epsilon k] \leq \frac{MGF(Z)}{e^{\lambda(k+\epsilon k)}} &\leq \frac{(1+\epsilon)^{\frac{k}{2}}}{e^{\lambda(k+\epsilon k)}} \\
&= \frac{(1+\epsilon)^{\frac{k}{2}}}{e^{\frac{\epsilon k}{2}}} \\
&= \left[ e^{ln(1+\epsilon)} \times e^{-\epsilon} \right]^{\frac{k}{2}} \\
&= \left[ e^{\epsilon - \frac{\epsilon^2}{2} + c \epsilon^3} \times e^{-\epsilon} \right]^{\frac{k}{2}}  \text{(Taylor series expansion)}\\
&= e^{-k(\epsilon^2 - d \epsilon^3)}
\end{split}
\end{equation}
Where $d$ is some constant. We can similarly prove $Pr[-(Z-k)>\epsilon k] \leq e^{-k(\epsilon^2 - d \epsilon^3)}$. Therefore, $Pr[|Z-k|\geq \epsilon k] \leq 2 e^{-k(\epsilon^2 - d\epsilon^3)}$ for some $d>0$.
\end{proof}

\subsection{Proving JL-Lemma}
\label{sec:jllemmaproof}
Recall that Lemma ~\ref{lemma:jllemma} states that if $A$ is a $k\times m$ matrix whose elements are independently drawn from $N(0,1)$ then $\forall x\in \mathbb{R}^m$, $P\left[\norm{Ax}_2\in (1+\epsilon) \norm{x}_2 \right]>1-\delta$ when $k=O\left(log(\frac{1}{\delta})\frac{1}{\epsilon^2}\right)$. We can write $Ax$ as: \\
\[
\begin{bmatrix}
 A_{11} & A_{12} & \cdots & A_{1m} \\
 \vdots & \vdots & \vdots & \vdots \\
 A_{k1} & A_{k2} & \cdots & A_{km} 
\end{bmatrix}
\begin{bmatrix}
 x_1 \\
 \vdots\\
 x_m
\end{bmatrix}
=
\begin{bmatrix}
 c_1 \\
 \vdots\\
 c_k
\end{bmatrix}
\]
We will prove for $\norm{x}_2=1$. Observe that $c_1=A_{11}x_1 + A_{12}x_2... + A_{1m}x_m$. $c_1\sim N(0,x_1^2) + N(0,x_2^2) + ... + N(0,x_m^2)$. Therefore, $c_1\sim N(0, x_1^2 + x_2^2 + ... + x_m^2)=N(0,1)$. Let $\norm{Ax}_2^2=Z=c_1^2+c_2^2+...+c_k^2$. We have:
\begin{equation}
\begin{split}
Pr\left[\left| \frac{\norm{Ax}_2^2}{k} - 1 \right| \geq \epsilon \right] &= Pr\left[|Z-k| \geq \epsilon k\right] \\
&\leq 2^{\frac{-k(\epsilon^2-d\epsilon^3)}{2}} \\
&\leq \delta
\end{split}
\end{equation}
The last inequality is true when $k=O\left(log(\frac{1}{\delta})\frac{1}{\epsilon^2}\right)$. This completes the proof.

\end{document}
